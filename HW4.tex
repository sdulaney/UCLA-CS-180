% Stewart Dulaney
% https://www.stewartdulaney.com
% This document was adapted from the templates posted at the following sources:
% https://www.cs.cmu.edu/~ckingsf/class/02-714/hw-template.tex
% http://www.math-cs.gordon.edu/courses/mat231/handouts/truth-table-latex.tex
%
\documentclass[11pt]{article}
\usepackage{amsmath,amssymb,amsthm,mathabx}
\usepackage{graphicx}
\usepackage[margin=1in]{geometry}
\usepackage{fancyhdr}
\usepackage{listings}
\lstset{
  basicstyle=\ttfamily,
  mathescape
}
\setlength{\parindent}{0pt}
\setlength{\parskip}{5pt plus 1pt}
\setlength{\headheight}{13.6pt}
\newcommand\question[2]{\vspace{.25in}\hrule\textbf{#1: #2}\vspace{.5em}\hrule\vspace{.10in}}
\renewcommand\part[1]{\vspace{.10in}\textbf{(#1)}}
\newcommand\answer{\vspace{.10in}\textbf{Answer: }}
\pagestyle{fancyplain}
\lhead{\textbf{\NAME\ (SID: \SID)}}
\chead{\textbf{HW\HWNUM}}
\rhead{CS 180, \today}
\begin{document}\raggedright
%Section A==============Change the values below to match your information==================
\newcommand\NAME{Stewart Dulaney}  % your name
\newcommand\SID{904-064-791}     % your ucla student id
\newcommand\HWNUM{4}              % the homework number
%Section B==============Put your answers to the questions below here=======================

\question{4.22}{}

\answer

We can provide a counterexample to show that $T$ itself is not necessarily a minimum-cost spanning tree (MST). Suppose we have an undirected graph $G = (V, E)$ with four vertices $v_1, v_2, v_3, v_4$ and edges $e_1 = \{v_1, v_3\}, e_2 = \{v_1, v_4\}, e_3 = \{v_2, v_3\}, e_4 = \{v_2, v_4\}, e_5 = \{v_3, v_4\}$. All edges have cost 10 except for $e_3$, which has cost 5.\newline

Consider the case when $T = \{e_1, e_5, e_4\}$. Each edge $e \in T$ belongs to an MST. Namely, the MST's $\{e_1, e_2, e_3\}, \{e_3, e_4, e_2\}, \{e_5, e_3, e_2\}$, which each have a cost of 25. However, $T$ has a cost of 30 and hence is not an MST. QED.

\clearpage

\question{4.25}{}

\answer

The given set of points $P = \{p_1, p_2, ..., p_n\}$ and distance function $d$ on the set $P$ form an undirected graph $G = (P, E)$ that is both weighted and complete. Each edge $e_{ij} = \{p_i, p_j\} \in E$ has weight $w = d(p_i, p_j) > 0$. The algorithm to build a hierarchical metric $\tau$ on $P$ can be described as follows.\newline

Let $T$ be the tree associated with $\tau$. We place each of the $n$ points in $P$ at leaf nodes $v_i$ in $T$. We run Kruskal's algorithm to find an MST on $G$, inserting edges from $E$ in order of increasing weight as long as it does not create a cycle. Each time we insert an edge $\{p_x, p_y\}$ of weight $w$ in Kruskal's algorithm, we also create a new node $v_{xy}$ in $T$, make it the parent of $v_x$ (associated with $p_x$) and $v_y$ (associated with $p_y$), and assign $v_{xy}$ a height of $h_{v_{xy}} = w$. We continue this process to build $T$ from the bottom level up. When we insert an edge of weight $w$ that merges components $C_a$ and $C_b$ in Kruskal's algorithm, we also create a new node $v_{ab}$ in $T$, make it the parent of the subtrees that (by induction) consist of $C_a$ and $C_b$, and assign $v_{ab}$ a height of $h_{v_{ab}} = w$.\newline

(i) $\tau$ is consistent with $d$\\

We can prove that for all pairs $i$, $j$ we have $\tau(p_i, p_j) \leq d(p_i, p_j)$. Notice that for any $p_i$ and $p_j$, $h_v = \tau(p_i, p_j)$ is equal to the weight of the edge that first merged the components containing $p_i$ and $p_j$ when Kruskal's algorithm was run on $G$. If the direct edge $\{p_i, p_j\}$ was considered before $p_i$ and $p_j$ were merged into the same component, then $\tau(p_i, p_j) = d(p_i, p_j)$. If $p_i$ and $p_j$ were merged into one component with an edge of weight $w$ in Kruskal's algorithm before the direct edge was considered, then $w = \tau(p_i, p_j) \leq d(p_i, p_j)$ because Kruskal's algorithmoperates in order of increasing weight.

(ii) if $\tau'$ is any other hierarchical metric consistent with $d$, then $\tau'(p_i, p_j) \leq \tau(p_i, p_j)$ for each pair of points $p_i$ and $p_j$.\\

We can prove this by contradiction. Assume $\tau'$ is another hierarchical metric such that $\tau'(p_i, p_j) > \tau(p_i, p_j)$. Let $T'$ be the tree associated with $\tau'$, $v'$ be the least common ancestor (LCA) of $p_i$ and $p_j$ in $T'$, and $T_i'$ and $T_j'$ be the subtrees below $v'$ containing the nodes associated with $p_i$ and $p_j$. Let ${h'}_{v'}$ be the height of $v'$ in $T'$ and by our assumption $\tau'(p_i, p_j) = {h'}_{v'} > \tau(p_i, p_j)$.\\

Notice there is a path $M$ from $p_i$ to $p_j$ in the MST we constructed from G in our algorithm above because a tree is connected by definition. Because $p_j$ is not in the subtree $T_i'$, $M$ must contain a point that is not in $T_i'$. Let $m$ be the first point along $M$ not in $T_i'$, and $m'$ be the point directly before $m$ on $M$ ($m'$ is a point in $T_i'$). Then $d(m, m') \geq {h'}_{v'} > \tau(p_i, p_j)$ by our assumption and because the least common ancestor (LCA) of $m$ and $m'$ must have height greater than or equal to ${h'}_{v'}$. However, all edges of $M$ were already present when Kruskal's algorithm merged the components containing $p_i$ and $p_j$. So each edge must have length less than or equal to $\tau(p_i, p_j)$ and we have our contradiction as desired. $\Rightarrow \Leftarrow$.\newline

Finally, note that this is a polynomial-time algorithm. We used Kruskal's algorithm to construct $T$, so it has running time $O(m logn)$ on a graph with $n$ nodes and $m$ edges.

\clearpage

\question{4.28}{}

\answer

The criteria for a valid solution for CluNet is a spanning tree $T$ of $G$ with exactly $k$ $X$-edges (the remaining edges will be owned by $Y$). The algorithm can be described as follows. First, we assign all $X$-edges to have weight 1 and $Y$-edges to have weight 10. Then we run a minimum spanning tree algorithm, which yields a spanning tree $T_i$ that contains the maximum number of $X$-edges (due to the weights). Let $i$ be the maximum number of $X$-edges. Next, we assign all $X$-edges to have weight 10 and $Y$-edges to have weight 1. Again, we run a minimum spanning tree algorithm, which this time yields a spanning tree $T_j$ that contains the minimum number of $X$-edges. Let $j$ be the minimum number of $X$-edges.\newline

Case 1:\\
If $k < j$ or $k > i$, we return false because there is no valid solution.\newline

Case 2:\\
If $k = j$, we return the spanning tree with the minimum number of $X$-edges.\newline

Case 3:\\
If $k = i$, we return the spanning tree with the maximum number of $X$-edges.\newline

Case 4:\\
If $j < k < i$, we can find the spanning tree with exactly $k$ $X$-edges and return it. We do this by adding a $X$-edge to $T_j$ from $T_i$ to form a cycle whose edges are not all $X$-edges. Then we delete a $Y$-edge and we have found $T_{j+1}$. We repeat this process until we find $T_k$, then return it.\newline

Note this is a polynomial-time algorithm. The two passes through a minimum spanning tree algorithm take $2 * n^2$ steps in the worst case, depending on implementation details. Cases 1 - 3 run in linear time because we need to count the number of $X$-edges and Case 4 takes at most $n^2$ steps to search for where to insert and delete edges. Hence, the total running time is $O(2n^2 + n + n^2) = O(3n^2 + n) = O(n^2)$.

\clearpage

\question{4.30}{}

\answer

The problem stipulates that $X \subseteq V$ is the set of $k$ terminals that must be connected by edges and hence $|X| = k$. Let $Y = Z - X$, so $Y$ is the set of non-terminal nodes in the minimum-weight Steiner tree $T$ on $X \cup Z \subseteq V$. Notice that any node $v$ in $Y$ must have degree greater than 2 in $T$. If $v$ had 1 edge, we should remove $v$ from $Y$ to get a lower weight Steiner tree. If $v$ had 2 edges, again we should remove $v$ from $Y$, this time adding an edge between the two neighbors of $v$. This yields a new Steiner tree with weight less than or equal to the original Steiner tree because the weights on $G$ satisfy the triangle inequality $w_{ik} \leq w_{ij} + w_{jk}$. Let $s$ be the number of nodes in our Steiner tree. Because the sum of the degrees in the tree is $2s - 2$, we know that the number of leaves is at least the number of nodes of degree greater than 2. Therefore $|Z| \leq k$ and we know how to find the minimum-weight Steiner tree. We can compute the MST on all sets $X \cup Z$ with $|Z| \leq k$ and the one with the lowest weight will be the minimum-weight Steiner tree. There is a maximum of ${n \choose 2k}$ such sets to check, so the running time of the algorithm to find a minimum-weight Steiner tree on $X$ will be $O(n^{O(k)})$, as desired.

\clearpage

\end{document}
