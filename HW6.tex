% Stewart Dulaney
% https://www.stewartdulaney.com
% This document was adapted from the templates posted at the following sources:
% https://www.cs.cmu.edu/~ckingsf/class/02-714/hw-template.tex
% http://www.math-cs.gordon.edu/courses/mat231/handouts/truth-table-latex.tex
%
\documentclass[11pt]{article}
\usepackage{amsmath,amssymb,amsthm,mathabx}
\usepackage{graphicx}
\usepackage[margin=1in]{geometry}
\usepackage{fancyhdr}
\usepackage{listings}
\lstset{
  basicstyle=\ttfamily,
  mathescape,
  numbers=left, 
  numberstyle=\small, 
  stepnumber=1, 
  numbersep=15pt
}
\setlength{\parindent}{0pt}
\setlength{\parskip}{5pt plus 1pt}
\setlength{\headheight}{13.6pt}
\newcommand\question[2]{\vspace{.25in}\hrule\textbf{#1: #2}\vspace{.5em}\hrule\vspace{.10in}}
\renewcommand\part[1]{\vspace{.10in}\textbf{(#1)}}
\newcommand\answer{\vspace{.10in}\textbf{Answer: }}
\pagestyle{fancyplain}
\lhead{\textbf{\NAME\ (SID: \SID)}}
\chead{\textbf{HW\HWNUM}}
\rhead{CS 180, \today}
\begin{document}\raggedright
%Section A==============Change the values below to match your information==================
\newcommand\NAME{Stewart Dulaney}  % your name
\newcommand\SID{904-064-791}     % your ucla student id
\newcommand\HWNUM{6}              % the homework number
%Section B==============Put your answers to the questions below here=======================

\question{6.19}{}

\answer

We will use dynamic programming.\newline

Let the binary string $s$ have $n$ characters. Let $x'$ be the repetition of $x$ with $n$ characters. Let $y'$ be the repetition of $y$ with $n$ characters. We rephrase the problem as: is $s$ an interleaving of $x'$ and $y'$? This way we don't have to deal with more than one pass through $x'$ or $y'$ because they are both already as long as $s$. Let $s[j]$ be the $jth$ character in $s$ and $s[1 : j]$ be the slice of the first $j$ characters in $s$. WLOG we define the same notation for $x'$ and $y'$.\newline

It's given that if $s$ is an interleaving of $x'$ and $y'$, then the last character in $s$ comes from either $x'$ or $y'$. Therefore, by removing the last character, we can get a smaller recursive problem on $s[1 : n - 1]$ with the repetitions $x'$ and $y'$ (taking the prefix of whichever repetition contained the last character).

Subproblems:\\

Let $T(i, j)$ = yes if $s[1 : i + j]$ is an interleaving of $x'[1 : i]$ and $y'[1 : j]$.\newline

Recurrence:\\

Let $T(i, j)$ = yes iff $(T(i - 1, j)$ = yes AND $s[i + j] = x'[i])$ OR $(T(i, j - 1)$ = yes AND $s[i + j] = y'[j])$.\newline

Algorithm:\\

\begin{lstlisting}
Is-Interleaving($s$, $x'$, $y'$, $n$)
  Array $M[0...n][0...n]$
  $M[0][0]$ = yes
  for $i = 1,...,n$
    for $j = 1,...,n$
      if $i + j \leq n$
        if $M[i - 1][j]$ = yes AND $s[i + j] = x'[i]$
          $M[i][j]$ = yes
        if $M[i][j - 1]$ = yes AND $s[i + j] = y'[j]$
          $M[i][j]$ = yes
        if $i + j = n$ AND $M[i][j]$ = yes
          return yes
        $M[i][j]$ = no
  return no
\end{lstlisting}

Time complexity:\\

The outer for loop runs $n$ times and the inner for loop runs at most $n$ times, and the statements within the inner for loop run in constant time. Therefore, the total running time is $O(n * n * 1) = O(n^2)$.

\clearpage

\end{document}
